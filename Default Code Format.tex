\documentclass[
	12pt, % Default font size, values between 10pt-12pt are allowed
	%letterpaper, % Uncomment for US letter paper size
	%spanish, % Uncomment for Spanish
]{fphw}

% Template-specific packages
\usepackage[utf8]{inputenc} % Required for inputting international characters
\usepackage[T1]{fontenc} % Output font encoding for international characters
\usepackage{mathpazo} % Use the Palatino font

\usepackage{graphicx} % Required for including images

\usepackage{booktabs} % Required for better horizontal rules in tables

\usepackage{listings} % Required for insertion of code \usepackage{enumerate} % To modify the enumerate environment

\usepackage{graphicx}
\usepackage{listings}
\usepackage[dvipsnames]{xcolor} %frame color
\usepackage{framed} % for defining framed environment
%----------------------------------------------------------------------------------------
%	Format
%----------------------------------------------------------------------------------------
\newenvironment{formal}{
 \def\FrameCommand{{\color{YellowOrange}\vrule width 2pt}\hspace{2pt}}
 \MakeFramed{\advance\hsize-\width}
 \vspace{2pt}\noindent\hspace{-7pt}\vspace{3pt}
}{\vspace{3pt}\endMakeFramed}
\definecolor{codegreen}{rgb}{0,0.6,0}
\definecolor{codegray}{rgb}{0.5,0.5,0.5}
\definecolor{codepurple}{rgb}{0.58,0,0.82}
\definecolor{backcolour}{rgb}{0.95,0.95,0.92}

\lstdefinestyle{mystyle}{
    backgroundcolor=\color{backcolour},   
    commentstyle=\color{codegreen},
    keywordstyle=\color{magenta},
    numberstyle=\tiny\color{codegray},
    stringstyle=\color{codepurple},
    basicstyle=\ttfamily\footnotesize,
    breakatwhitespace=false,         
    breaklines=true,                 
    captionpos=b,                    
    keepspaces=true,                 
    numbers=left,                    
    numbersep=5pt,                  
    showspaces=false,                
    showstringspaces=false,
    showtabs=false,                  
    tabsize=2
}

\lstset{style=mystyle}
%----------------------------------------------------------------------------------------
%	ASSIGNMENT INFORMATION
%----------------------------------------------------------------------------------------

\title{Assignment \#3} % Assignment title

\author{Eqan Ahmad\\
  19F0256
} % Student name


\institute{National University \\ Of Computer Science And Technology} % Institute or school name

\class{Data Structures(CS 218)} % Course or class name

\professor{Dr Hashim Yasin} % Professor or teacher in charge of the assignment

%----------------------------------------------------------------------------------------
%	Document Content
%----------------------------------------------------------------------------------------
\begin{document}
\maketitle{}
\section*{Question 1 }
\subsection*{\centering{Terminal Color Scheme}}
\lstinputlisting[language=C++]{colormod.h}
\subsection*{\centering{C++ Custom Shell}}
\lstinputlisting[language=C++]{C++Shell.h}
\clearpage
\subsection*{\centering{Header File}}
\lstinputlisting[language=C++]{DSCourse3-Q1.h}
\clearpage
\subsection*{\centering{CPP File}}
\lstinputlisting[language=C++]{DSCourse3-Q1v2.cpp}
\clearpage
\subsection*{\centering{Main File}}
\lstinputlisting[language=C++]{DSCourse3-Q1Main.cpp}
 \begin{formal}
 {\bf Note}
	This program is specifically made for Unix/Linux command line, As a linux User i wanted to integrate shell like commands into my
	 program and my drive was enough to combine some fresh colorschemes with Bash Like commands into this program.\\
 \end{formal}
\begin{figure}[!htb]
  \includegraphics[width=\linewidth]{1.png}
  \caption{ScreenShot}
\end{figure}
\begin{figure}[!htb]
  \includegraphics[width=\linewidth]{2.png}
  \caption{ScreenShot}
\end{figure}
\begin{figure}[!htb]
  \includegraphics[width=\linewidth]{3.png}
  \caption{ScreenShot}
\end{figure}
\begin{figure}[!htb]
  \includegraphics[width=\linewidth]{4.png}
  \caption{ScreenShot}
\end{figure}
\begin{figure}[!htb]
  \includegraphics[width=\linewidth]{5.png}
  \caption{ScreenShot}
\end{figure}
\begin{figure}[!htb]
  \includegraphics[width=\linewidth]{6.png}
  \caption{ScreenShot}
\end{figure}
\begin{figure}[!htb]
  \includegraphics[width=\linewidth]{7.png}
  \caption{ScreenShot}
\end{figure}
\begin{figure}[!htb]
  \includegraphics[width=\linewidth]{8.png}
  \caption{ScreenShot}
\end{figure}
\begin{figure}[!htb]
  \includegraphics[width=\linewidth]{9.png}
  \caption{ScreenShot}
\end{figure}
\begin{figure}[!htb]
  \includegraphics[width=\linewidth]{10.png}
  \caption{ScreenShot}
\end{figure}
\begin{figure}[!htb]
  \includegraphics[width=\linewidth]{11.png}
  \caption{ScreenShot}
\end{figure}
\begin{figure}[!htb]
  \includegraphics[width=\linewidth]{12.png}
  \caption{ScreenShot}
\end{figure}
\begin{figure}[!htb]
  \includegraphics[width=\linewidth]{13.png}
  \caption{ScreenShot}
\end{figure}
\clearpage
\end{document}
